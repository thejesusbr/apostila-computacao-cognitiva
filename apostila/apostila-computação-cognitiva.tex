\documentclass[a4paper,12pt,oneside]{book}

\usepackage[T1]{fontenc}
\usepackage[brazilian]{babel}
\usepackage{apostila}
\usepackage{indentfirst}
\usepackage{etoc}
\usepackage{lettrine}
\hypersetup{%
   bookmarks=,%
   bookmarksnumbered=,%
   pdfauthor=Dr. Wendell F. S. Diniz%
}
\title{Computação Cognitiva}
\author{Wendell F. S. Diniz}
\date{01/01/20}
\companion{https://github.com/thejesusbr/apostila-computacao-cognitiva}

\renewcommand{\baselinestretch}{1.5}



\newcommand*\chaptertoc{%
  \setcounter{tocdepth}{3}%
  \renewcommand{\etocbkgcolorcmd}{\color{unisblue!20}}
  \renewcommand{\etocleftrulecolorcmd}{\color{unisblue!20}}
  \renewcommand{\etocrightrulecolorcmd}{\color{unisblue!20}}
  \renewcommand{\etocbottomrulecolorcmd}{\color{unisblue!20}}
  \renewcommand{\etoctoprulecolorcmd}{\color{unisblue!20}}
  \renewcommand{\etocbelowtocskip}{0pt\relax}
  \etocframedstyle [1]{%
    %\fbox
  }%
  \localtableofcontents
  \vspace{1pt}
}%

\begin{document}

\frontmatter
\pagestyle{empty}
\maketitle
\newpage
\renewcommand{\etocbkgcolorcmd}{\relax}
\renewcommand{\etocleftrulecolorcmd}{\color{white}}
\renewcommand{\etocrightrulecolorcmd}{\color{white}}
\renewcommand{\etocbottomrulecolorcmd}{\color{white}}
\renewcommand{\etoctoprulecolorcmd}{\color{white}}
 \etocframedstyle [1]{%
    %\fbox
  }%
\tableofcontents

\mainmatter

\pagestyle{fancy}
 
\chapter{Introdução}

  \chaptertoc

  \lettrine{Q}{uando} falamos em Computação Cognitiva, então, estamos falando de uma maneira de criar programas que tomam decisões de modo similar ao processo do raciocínio humano...
  
  Uma das primeiras noções...
  
  \section{Inteligência Artificial}
  
  Inteligência Artificial, ou IA...
  
  \section{Aprendizado de Máquina}
  
  Aprendizado de Máquina (ML)...
 
\end{document}
